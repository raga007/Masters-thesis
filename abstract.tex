Korat is an existing technique for test input generation using imperative constraints that describe properties of desired inputs written as Java predicates, termed repOk methods, which are executable checks for those properties.  Korat efficiently prunes the space of candidate inputs for the repOk method by executing it on candidate inputs and monitoring the object fields that repOk accesses in deciding if the properties are satisfied.  While Korat generates inputs effectively, its correctness and efficiency rely on two assumptions about the repOk methods.  For correctness, Korat assumes the repOk methods do not use the Java reflection API for field accesses; the use of reflection renders Korat unable to enumerate all desired inputs.  For efficiency, Korat assumes the repOk methods are written such that they only access fields that are necessary to check the properties; the use of unnecessary field accesses by repOk renders Korat's pruning ineffective.  Our thesis addresses both these limitations.  To support reflection, we build on the core Korat to enhance it such that it can monitor field accesses based on reflection.  To deal with unnecessary field accesses, we introduce a static analysis tool that points out unnecessary filed accesses in repOks. Experimental results using a suite of standard data structure subjects show the effectiveness of our approach.
