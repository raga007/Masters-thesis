\chapter{Experiments}
\label{ch:evaluation}
To demonstrate the potential benefits of the two techniques we
introduced in Chapters~\ref{ch:adding-reflection} and~\ref{ch:static-analysis}, we
next present experimental results that show how the behavior of the
standard Korat technique~\cite{boyapati2002korat} is affected when
(1)~reflection is used for accessing fields in \emph{RepOk}
(Section~\ref{sec:generation-of-valid-structures}); and (2)~\emph{RepOk} makes extraneous field
accesses (Section~\ref{sec:performance-of-inefficient-predicates}).  As subjects, we use four
standard data structures (Section~\ref{sec:benchmarks}).
We performed all experiments on a Macbook running OSX with a 2.7GHz,
Intel core i7 processor using Sun’s Java 6 SDK 1.6.0 JVM.


\section{Benchmarks}
\label{sec:benchmarks}
The following list summarizes our subject data structures: 
\begin{itemize}
\item BinaryTree -- acyclic, binary trees with size field;
\item RedBlackTree -- acyclic, binary search trees with size field and
  (red/black) colored nodes.
\item SinglyLinkedList -- acyclic, singly-linked list with size field;
\item SortedList -- acyclic, singly-linked list with size field and
  integer elements in sorted order.
\end{itemize}

Table~\ref{fig:benchmarksAndFinitizationParams} lists the finitization parameters for each subject.

\begin{table}[h]
\resizebox{\columnwidth}{!}{\begin{tabular}{|c|c|}
\hline
\textit{Benchmark}    & \textit{Finitization parameters}       \\ \hline
\textit{BinaryTree}   &  \texttt{numNodes, minSize, maxSize}             \\ \hline
\textit{RedBlackTree} &  \texttt{numEntries, minSize, maxSize, numKeys}  \\ \hline
\textit{SinglyLinkedList}   &  \texttt{numEntries, minSize, maxSize}           \\ \hline
\textit{SortedList}   &  \texttt{numEntries, minSize, maxSize, numElems} \\ \hline
\end{tabular}}
\caption{Benchmarks and finitization parameters. Each benchmark is
  named after the class for which data structures are generated; the
  structures also contain objects from other classes.}
\label{fig:benchmarksAndFinitizationParams}
\end{table}

For each benchmark, we set all finitization parameters such that Korat
generates structures of upto the given \emph{size}. We set the
\emph{maxSize} and other maximum parameters to the value of
\emph{size} (also known as \emph{scope}) and \emph{minSize} to 0.

\section{Generation of valid structures}
\label{sec:generation-of-valid-structures}
\noindent The following list describes the various kinds of \emph{RepOk}
methods we use for evaluation:
\begin{itemize}
\item No reflection -- standard \emph{RepOk} implementation (from the
  Korat distribution), which does not use the Java reflection API for
  any of the field accesses;
\item Reflection for \emph{root/header} field access -- \emph{RepOk}
  uses reflection only to access the \emph{root/header} field, which
  is typically accessed during the initial part of a \emph{RepOk's}
  execution;
\item Reflection for \emph{size} field access -- \emph{RepOk} uses
  reflection only to access the \emph{size} field, which is typically
  accessed during the final part of a \emph{RepOk's} execution; and
\item Reflection for a \emph{recursive} field access -- \emph{RepOk}
  uses reflection only to access a \emph{recursive} field, which is
  typically accessed multiple times by \emph{RepOk}.
\end{itemize}

For a range of sizes and four different \emph{RepOk} implementations,
we tabulate the number of candidate structures that Korat evaluates,
the number of valid structures that Korat generates.
Table~\ref{fig:resultsTable} presents the results, which show that
Korat behaves quite differently when reflection is used even for a
single field.  For example, for binary tree, for size 5, when
reflection is used for just the \texttt{size} field, the number of
valid trees generated reduces to 6 (from 65).  In all cases, the
number of valid structures produced is reduced significantly; thus,
many valid structures are no longer being generated, which is as
expected since the Korat search is no longer able to create the search
space correctly based on the field that Korat can no longer monitor.
Likewise, the number of structures explored is also reduced when
reflection is used.

Using our technique to support reflection
(Chapter~\ref{ch:adding-reflection}), the standard Korat algorithm is
able to generate exactly the expected number of valid structures and
also explore exactly the expected number of candidate structures in
each case as the original \emph{RepOk} that does not use reflection.

\begin{table}[h]
\centering
\resizebox{\columnwidth}{!}{\begin{tabular}{|c|c|c|c|c|c|c|c|c|c|c|c|c|}
\hline
 & \multicolumn{3}{c|}{\textit{\begin{tabular}[c]{@{}c@{}}No reflection - \# of \\ structures\\ explored/valid\end{tabular}}} & \multicolumn{3}{c|}{\textit{\begin{tabular}[c]{@{}c@{}}Reflection for \\ ``root/header'' field \\ access - \# of structures\\ explored/valid\end{tabular}}} & \multicolumn{3}{c|}{\textit{\begin{tabular}[c]{@{}c@{}}Reflection for\\  ``size'' field access - \# of\\  structures\\ explored/valid\end{tabular}}} & \multicolumn{3}{c|}{\textit{\begin{tabular}[c]{@{}c@{}}Reflection for a \\ recursive field \\ access - \# of structures\\ explored/valid\end{tabular}}} \\ \hline
\textit{Scope size} & 5 & 6 & 7 & 5 & 6 & 7 & 5 & 6 & 7 & 5 & 6 & 7 \\ \hline
\textit{BinaryTree} & \begin{tabular}[c]{@{}c@{}}1272/\\ 65\end{tabular} & \begin{tabular}[c]{@{}c@{}}4835/\\ 197\end{tabular} & \begin{tabular}[c]{@{}c@{}}18474/\\ 626\end{tabular} & \begin{tabular}[c]{@{}c@{}}7/\\ 2\end{tabular} & \begin{tabular}[c]{@{}c@{}}8/\\ 2\end{tabular} & \begin{tabular}[c]{@{}c@{}}9/\\ 2\end{tabular} & \begin{tabular}[c]{@{}c@{}}952/\\ 6\end{tabular} & \begin{tabular}[c]{@{}c@{}}3659/\\ 7\end{tabular} & \begin{tabular}[c]{@{}c@{}}14099/\\ 8\end{tabular} & \begin{tabular}[c]{@{}c@{}}9/\\ 2\end{tabular} & \begin{tabular}[c]{@{}c@{}}10/\\ 2\end{tabular} & \begin{tabular}[c]{@{}c@{}}11/\\ 2\end{tabular} \\ \hline
\textit{RedBlackTree} & \begin{tabular}[c]{@{}c@{}}6073/\\ 115\end{tabular} & \begin{tabular}[c]{@{}c@{}}25938/\\ 327\end{tabular} & \begin{tabular}[c]{@{}c@{}}112012/\\ 911\end{tabular} & \begin{tabular}[c]{@{}c@{}}9/\\ 2\end{tabular} & \begin{tabular}[c]{@{}c@{}}10/\\ 2\end{tabular} & \begin{tabular}[c]{@{}c@{}}11/\\ 2\end{tabular} & \begin{tabular}[c]{@{}c@{}}1387/\\ 6\end{tabular} & \begin{tabular}[c]{@{}c@{}}4958/\\ 7\end{tabular} & \begin{tabular}[c]{@{}c@{}}18769/\\ 8\end{tabular} & \begin{tabular}[c]{@{}c@{}}16/\\ 6\end{tabular} & \begin{tabular}[c]{@{}c@{}}18/\\ 7\end{tabular} & \begin{tabular}[c]{@{}c@{}}20/\\ 8\end{tabular} \\ \hline
\textit{SinglyLinkedList} & \begin{tabular}[c]{@{}c@{}}51/\\ 6\end{tabular} & \begin{tabular}[c]{@{}c@{}}70/\\ 7\end{tabular} & \begin{tabular}[c]{@{}c@{}}92/\\ 8\end{tabular} & \begin{tabular}[c]{@{}c@{}}7/\\ 2\end{tabular} & \begin{tabular}[c]{@{}c@{}}8/\\ 2\end{tabular} & \begin{tabular}[c]{@{}c@{}}9/\\ 2\end{tabular} & \begin{tabular}[c]{@{}c@{}}26/\\ 1\end{tabular} & \begin{tabular}[c]{@{}c@{}}34/\\ 1\end{tabular} & \begin{tabular}[c]{@{}c@{}}43/\\ 1\end{tabular} & \begin{tabular}[c]{@{}c@{}}9/\\ 2\end{tabular} & \begin{tabular}[c]{@{}c@{}}10/\\ 2\end{tabular} & \begin{tabular}[c]{@{}c@{}}11/\\ 2\end{tabular} \\ \hline
\textit{SortedList} & \begin{tabular}[c]{@{}c@{}}891/\\ 252\end{tabular} & \begin{tabular}[c]{@{}c@{}}4030/\\ 924\end{tabular} & \begin{tabular}[c]{@{}c@{}}18110/\\ 3432\end{tabular} & \begin{tabular}[c]{@{}c@{}}11/\\ 6\end{tabular} & \begin{tabular}[c]{@{}c@{}}13/\\ 7\end{tabular} & \begin{tabular}[c]{@{}c@{}}15/\\ 8\end{tabular} & \begin{tabular}[c]{@{}c@{}}26/\\ 1\end{tabular} & \begin{tabular}[c]{@{}c@{}}34/\\ 1\end{tabular} & \begin{tabular}[c]{@{}c@{}}43/\\ 1\end{tabular} & \begin{tabular}[c]{@{}c@{}}13/\\ 6\end{tabular} & \begin{tabular}[c]{@{}c@{}}15/\\ 7\end{tabular} & \begin{tabular}[c]{@{}c@{}}17/\\ 8\end{tabular} \\ \hline
\end{tabular}}
\caption{Number of valid structures generated by Korat in different scenarios of using reflection.}
\label{fig:resultsTable}
\end{table}




\section{Performance of Inefficient predicates}
\label{sec:performance-of-inefficient-predicates}
This section describes how poorly written \emph{RepOk's} can influence
Korat's performance.  Specifically, for each of our 4 subject data
structures, we create a \emph{RepOk} that does not return its result as soon
it is determined (e.g., when a property violation is established) and
continues to make extraneous field accesses.  All these poorly written
\emph{RepOk's} follow the structure of the \emph{RepOk} shown in
Figure~\ref{fig:repOkMultipleBooleanVariables}, where a boolean field
is used to keep track of each property and the field is set to false
when the property is found to be false instead of returning false
immediately.

% This table has been revised and every value has been verified
\begin{table}[h]
\resizebox{\columnwidth}{!}{\begin{tabular}{|c|c|c|c|c|c|c|c|c|c|c|c|c|}
\hline
 & \multicolumn{3}{c|}{\textit{\begin{tabular}[c]{@{}c@{}}\# of structures \\ explored/valid \\ with an \\ short-circuiting RepOk\end{tabular}}} & \multicolumn{3}{c|}{\textit{\begin{tabular}[c]{@{}c@{}}Execution time\\ (in sec)\end{tabular}}} & \multicolumn{3}{c|}{\begin{tabular}[c]{@{}c@{}}\# of structures \\ explored/valid \\ with an \\ poorly-written RepOK\end{tabular}} & \multicolumn{3}{c|}{\textit{\begin{tabular}[c]{@{}c@{}}Execution time\\ (in sec)\end{tabular}}} \\ \hline
\textit{Scope size} & 4 & 5 & 6 & 4 & 5 & 6 & 4 & 5 & 6 & 4 & 5 & 6 \\ \hline
\textit{BinaryTree} & \begin{tabular}[c]{@{}c@{}}337/\\ 23\end{tabular} & \begin{tabular}[c]{@{}c@{}}1272/\\ 65\end{tabular} & \begin{tabular}[c]{@{}c@{}}4835/\\ 197\end{tabular} & 0.21 & 0.24 & 0.35 & \begin{tabular}[c]{@{}c@{}}105455/\\ 23\end{tabular} & \begin{tabular}[c]{@{}c@{}}3922746/\\ 65\end{tabular} & \begin{tabular}[c]{@{}c@{}}171432317/\\ 197\end{tabular} & 0.48 & 2.64 & 118.15 \\ \hline
\textit{LinkedList} & \begin{tabular}[c]{@{}c@{}}35/\\ 5\end{tabular} & \begin{tabular}[c]{@{}c@{}}51/\\ 6\end{tabular} & \begin{tabular}[c]{@{}c@{}}70/\\ 7\end{tabular} & 0.18 & 0.19 & 0.20 & \begin{tabular}[c]{@{}c@{}}75/\\ 5\end{tabular} & \begin{tabular}[c]{@{}c@{}}126/\\ 6\end{tabular} & \begin{tabular}[c]{@{}c@{}}196/\\ 7\end{tabular} & 0.18 & 0.19 & 0.19 \\ \hline
\textit{SortedList} & \begin{tabular}[c]{@{}c@{}}203/\\ 70\end{tabular} & \begin{tabular}[c]{@{}c@{}}891/\\ 252\end{tabular} & \begin{tabular}[c]{@{}c@{}}4030/\\ 924\end{tabular} & 0.21 & 0.24 & 0.30 & \begin{tabular}[c]{@{}c@{}}411/\\ 70\end{tabular} & \begin{tabular}[c]{@{}c@{}}4026/\\ 252\end{tabular} & \begin{tabular}[c]{@{}c@{}}56176/\\ 924\end{tabular} & 0.22 & 0.28 & 0.45 \\ \hline
\textit{RedBlackTree} & \begin{tabular}[c]{@{}c@{}}1270/\\ 41\end{tabular} & \begin{tabular}[c]{@{}c@{}}6073/\\ 115\end{tabular} & \begin{tabular}[c]{@{}c@{}}25938/\\ 327\end{tabular} & 0.38 & 0.50 & 0.58 & \begin{tabular}[c]{@{}c@{}}684967/\\ 41\end{tabular} & \begin{tabular}[c]{@{}c@{}}31247487/\\ 115\end{tabular} & \begin{tabular}[c]{@{}c@{}}1629892151/\\ 327\end{tabular} & 1.28 & 24.73 & 2251.12 \\ \hline
\end{tabular}}
\caption{Korat's performance degradation on poorly-written RepOk's. Time is elapsed real time in seconds.}
\label{fig:peformanceResultsTable}
\end{table}

Table~\ref{fig:peformanceResultsTable} tabulates the experimental
results.  In all cases, using the poorly-written \emph{RepOk}, Korat
generates exactly the expected number of valid structures, so the
output is correct, but the search space explored is substantially
larger than that using the short-circuiting \emph{RepOk}.  For
example, for binary tree with size 6, the number of structures
explored increases many-fold even though the number of valid
structures generated stays the same.  Likewise, the time to explore
using poorly-written \emph{RepOk's} is also significantly larger.
