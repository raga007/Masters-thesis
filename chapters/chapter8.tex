\chapter{Conclusion and Future work}
\label{ch:conclusion-and-future-work}

This chapter first summarizes our work (Section~\ref{sec:conclusion}) and
then outlines some ideas for future work (Section~\ref{sec:future-work}).

\section{Conclusion}
\label{sec:conclusion}
We developed two techniques to improve the previously developed Korat
tool for test input generation using imperative constraints that
describe properties of desired inputs written as Java predicates,
termed \emph{RepOk} methods.  A \emph{RepOk} is an executable check
for the desired properties.  Korat efficiently prunes the space of
candidate inputs for the \emph{RepOk} method by executing it on
candidate inputs and monitoring the object fields that \emph{RepOk}
accesses in deciding if the properties are satisfied.  While Korat
generates inputs effectively, its correctness and efficiency rely on
two assumptions about the repOk methods.  For correctness, Korat
assumes the \emph{RepOk} methods do not use the Java reflection API
for field accesses; the use of reflection renders Korat unable to
enumerate all desired inputs.  For efficiency, Korat assumes the
\emph{RepOk} methods are written such that they only access fields
that are necessary to check the properties; the use of unnecessary
field accesses by \emph{RepOk} renders Korat's pruning ineffective.
Our work addressed both these limitations.  To support reflection, we
built on the core Korat to enhance it such that it can monitor field
accesses based on reflection.  To deal with unnecessary field
accesses, we introduced a static analysis tool that points out
unnecessary filed accesses in \emph{RepOk} methods. Experimental
results using a suite of standard data structure subjects show the
effectiveness of our approach.


\section{Future work}
\label{sec:future-work}
Recall, Korat requires the user to provide a RepOk method and a
finitization.  We next discuss some ideas that may assist users in
using Korat more effectively.

\para \textbf{An integrated RepOkValidator.} The RepOkValidator
presented in Chapter~\ref{ch:static-analysis} is currently a
command-line utility.  It would be very useful to add a GUI front-end
and release it as a tool that integrates with Korat to help users
develop more effective RepOk methods.

\begin{comment}
\para \textbf{Reflection instrumentation.}
Chapter~\ref{ch:adding-reflection} presented our technique that builds
on Korat to support the usage of reflection; however, our current
protoype tool implementation requires manually transforming the calls
to Java reflection API to our own reflection API (which mimics the
Java reflection API but also notifies Korat of field accesses).  An
additional simple engineering step that replaces calls to Java
reflection API to our API is required for fully automatic support of
allowing the use of reflection for field accesses.
\end{comment}

\para \textbf{Context-based test generation.} Korat by default
generates all non-isomorphic structures up to a given bound provided
by the finitization provided by the user. One approach to reduce the
number of structures is to consider generation that may not be
exhaustive but is driven by the problem context. Doing so would make
Korat more efficient for generating higher quality tests. For example,
if the application under test is a web application, Korat can be
guided to generate only boundary tests (tests that have a higher
likelihood to cause a failure), as opposed to generating all tests.

\para \textbf{Continuous integration with Korat.} Future work can
adopt Korat in a continuous integration process where before a code
change is committed, Korat is run for automated testing, and the
change is allowed to commit only if no failures are reported, thereby
making bounded exhaustive testing a part of continuous
integration~\cite{fowler2006continuous}.
