\chapter{Related work}
\label{ch:related-work}
This chapter reviews the work related to improving Korat and static
analysis. Section \ref{sec:attempts-to-improve-korat} discusses other
projects that have improved different aspects of Korat. Section
\ref{sec:related-static-analyses} discusses a few static program
analysis tools that have tried to verify properties of data
structures.

\section{Attemps to improve Korat}
\label{sec:attempts-to-improve-korat}
Over the last decade, several projects have tried to improve various
aspects of Korat. Parallel test generation and execution with
Korat~\cite{misailovic2007parallel} presents algorithms to generate
test inputs with a parallel search in Korat and techniques to execute
test inputs in parallel, both off-line (when the inputs are saved on
disk) and on-line (when execution immediately follows
generation). PKorat~\cite{siddiqui2009pkorat} improves on this work,
explores the same state space as Korat but considers several
candidates in each iteration. These candidates are distributed among
parallel workers resulting in an efficient parallel version of
Korat. Parallel test generation and execution with Korat is designed
for clusters comprising largely independant machines and it randomly
divides work between these machines.  On the other hand, PKorat
systematically divides work for and it is primarily designed for high
bandwidth clusters.

The principle idea of Korat has been used in other applications. In
particular, STARC~\cite{elkarablieh2007starc} uses the Korat algorithm
to repair huge complex structures by running the algorithm in
neighborhood of the defective structure. Glass box
testing~\cite{darga2006efficient} uses the method to be tested to
prune Korat’s generation. Thus it moves away from the pure black-box
approach of Korat.

Elkarablieh introduced an efficient backtracking
optimization~\cite{elkarablieh2008efficient} that can undo operations
done in last execution and proceed from that point in the \emph{RepOk}
predicate for the next candidate.  This has shown improvements for
STARC and also for Korat. Siddiqui presented \emph{focused
  generation}~\cite{siddiqui2009optimizing}, a novel optimization in
Korat where the structure generation is focussed to some aspect of the
structure. For example, finding all structures with one data
assignment or finding all structures with one valid instance of a
previously tested contained structure. Later he introduced
Multi-Korat~\cite{siddiqui2012lightweight} uses a lightweight static
data-flow analysis and performs multi-value comparisions in the
candidate structure to \emph{forward} predicate executions on
candidate inputs that are similar. Multi-Korat is centered around
reducing the number of \emph{RepOk} executions to produce all test
cases.


\section{Static Analysis}
\label{sec:related-static-analyses}
Several projects have developed static analyses for verifying program
properties. We will discuss some of the projects that check properties
of complex data structures. The Extended Static Checker
(ESC)~\cite{flanagan2002extended} uses a theorem prover to verify
partial correctness of Java classes annotated with JML
specifications. ESC can find faults such as null pointer dereferences,
array bounds violations and division by zero. The Three-Valued-Logic
Analyzer (TVLA)~\cite{sagiv2002parametric} was the first system to
implement a shape analysis, which verifies that programs with
destructive updates preserve structural invariants of linked data
structures. There has been work~\cite{yorsh2004symbolically} that
extends TVLA to handle more properties of the content of the data
structures and uses symbolic computation to automatically derive
transfer functions, thus reducing the amount of work that the user
needs to do.

\para Kuncak et al. proposed Role Analysis~\cite{kuncak2002role}, a
compositional interprocedual analysis that verifies similar properties
like TVLA. The same authors also developed
Hob~\cite{lam2005generalized}, an analysis framework that combines
different static analysis and theorem proving
techniques~\cite{zee2004combining} to show complex data structure
consistency properties. The Pointer Assertion Logic Engine
(PALE)~\cite{moller2001pointer} can verify a large set of data
structures that have a spanning tree backbone, with possibly
additional pointers that do not add extra
information. Jalloy~\cite{jackson2000finding,vaziri2003checking}
analyzes methods that manipulate linked data structures by first
building an alloy model of Java code and then checking it in a
bounded-exhaustive manner with the Alloy Analyzer
\cite{jackson2000alcoa}.


