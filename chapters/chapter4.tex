\chapter{Static analysis to check predicate}
\label{ch:static-analysis}
This chapter presents a static analysis tool called
RepOkValidator. RepOkValidator checks if the \emph{RepOk} method
written by the user affects the output or the efficiency of the Korat
search. Section~\ref{sec:static-analysis-conditions} presents the
conditions that the \emph{RepOk} method needs to fulfill for the
Korat search to function efficiently and output the expected valid
structures. In Section~\ref{sec:static-analysis-conditions}, we
discuss why we chose static analysis over dynamic analysis to analyze
the \emph{RepOk} written by the user. Section~\ref{sec:repokvalidator}
introduces the RepOkValidtor and Section~\ref{sec:implementation}
discusses the implementation details of the RepOkValidator. Finally,
Section~\ref{sec:implementation} presents examples of poorly-written
\emph{RepOk} methods and the logs generated by the RepOkValidator for
these methods.


\section{Conditions}
\label{sec:static-analysis-conditions}
Recall from Chapter~\ref{ch:shortcomings-of-korat}, the execution of
Korat search is partly dependant on the predicate written by the
user. A \emph{RepOk} that doesn’t return the result as soon as
possible, may affect the number of structures pruned by Korat
search. Korat requires\cite{marinov2005automatic} that the following
conditions hold good for a user written \emph{RepOk} method -

\begin{aquote}{Darko Marinov}
C1 : Each execution of the predicate terminates, and either returns a
true or a false.\\ C2 : No execution of the predicate depends on the
actual allocation address of the candidate object or its fields.(This
holds for java predicates if no execution invokes
\emph{System.identityHashCode} method.)\\ C3 : Each field that the
predicate accesses is either a part of the candidate structure or a
field of some object that the predicate locally allocates. In other
words, the execution does not access global data through static
fields.\\
\end{aquote}

\para It is possible to make a guarantee about the efficiency of Korat
search (the number of candidate structures evaluated by Korat), if the
user written predicate follows the conditions below, in addition to
the conditions specified by Korat. In other words, the following
conditions only affect the efficiency of Korat search, unlike C1-C3
which affect the the output generated by Korat. The following
conditions assume that a fixed amount of space (variable length data
structures like boolean arrays/lists are not used) is allocated for
the booleans used by the predicate to store its result.
\begin{itemize}
\item C4 : If the return value of the predicate is a variable then, it
  is set at most once between the time of its declaration and its
  return along any path of execution.
\item C5 : If the return value is a conjunction or a disjunction of
  multiple boolean variables then, each variable is set at most once
  during the time between its declaration and usage in deciding the
  return value.
\item C6 : No fields are accessed (Chapter~\ref{ch:adding-reflection}
  describes a way to access the fields without Korat accounting for
  the access) between the time of setting the return value and the
  actual return of the value.
\end{itemize}

\para These conditions were derived from observing the results of
executions of Korat with various predicates that output the same
valid structures but evaluated different number of candidate
structures.

\section{Choosing static analysis over dynamic analysis}
\label{sec:choosing-static-over-dynamic}
Given a predicate, it is possible to use a static or dynamic analysis
of the predicate to derive detailed information about the
predicate. Dynamic analysis is the testing and evaluation of a piece of
code during runtime. Dynamic analysis reveals subtle faults whose
cause is often complex to be discovered by static analysis. It is
usually more time consuming than a static analysis. Static analysis is
the testing and evaluation of a piece of code by examining it, without
executing it. It examines all possible execution paths and variable
values, not just those invoked during execution.

\para From the conditions mentioned in section
\ref{sec:static-analysis-conditions}, the only condition that can’t be
checked using any analysis is C1, as it is the halting problem. Since
every other condition can be checked using a static analysis, we
decided to use static analysis over dynamic analysis in the interest
of avoiding the execution time overhead of dynamic analysis.

\section{RepOkValidator}
\label{sec:repokvalidator}
The RepOkValidator is a static analysis tool that will analyze the
predicate written by the user. It will print all the violations in the
form of a log. The log may help the user to ensure that the predicate
will help Korat search to output the expected valid structures without
evaluating excessive candidate structures. In other words, the
RepOkValidator will print violations of conditions C2-C6 and localize
them in the predicate to make it easier for the user to fix the
violations. It is important to understand that violations of type C2
and C3 need to be fixed before using the predicate with Korat as they
may affect the number of valid structures generated by Korat. On the
other hand, it is optional for the user to fix violations of type
C4-C6 as they just affect the efficiency of Korat search.

\para For each condition to hold good, the RepOkValidator needs to
look for specific violations of the conditions in the predicate.
\begin{itemize}
\item C2 : will be violated if the predicate written by the user
  invokes the System.identityHashCode method.
\item C3 : will be violated if any fields accessed in the predicate
  don’t belong to the candidate class or the fields accessed are
  static.
\item C4 : will be violated if the return value is assigned a value
  more than once, from the time it is defined to the time it is
  returned, in any path of execution.
\item C5 : will be violated if any of the variables that make up the
  result are assigned a value more than once from the time they are
  defined to the time they are returned, in any path of execution.
\item C6 : will be violated if any of the candidate class fields are
  accessed once all of the fields that make up the return value are
  assigned a value.
\end{itemize}

\para To be able to detect the violations, we build def-use chains of
all variables in the predicate. We also collect data about all method
invocations, including the parameters used to invoke the methods. Once
we have this data, the RepOkValidator traverses the def-use chains to
detect violations and logs them for user reference.

\section{Implementation}
\label{sec:implementation}
The implementation of RepOkValidator uses Soot~\cite{vallee1999soot}, a java
bytecode optimization framework. The framework is implemented in Java
and supports three intermediate representations of java
bytecode. RepOkValidator uses the intermediate representation called
\emph{Jimple} as it is an ideal representation for static analysis. Jimple is
a 3-address code representation of bytecode, which is typed and does
not include the 3 address code equivalent of a \emph{jsr} (jump to
subroutine) instructions.

\para The RepOkValidator takes the data structure that needs to be
analyzed as a command line argument and passes the argument on to
Soot, for it convert the class into Jimple. Once the code is converted
to Jimple, the RepOkValidator looks for all the methods that return a
\emph{boolean}, have a method name that starts with \emph{RepOk} and
take no arguments. For each such method, the RepOkValidator does
intraprocedural analysis to extract the def-use chains of all
variables in the method. Finally, the RepOkValidator validates the
predicate by checking for violations of conditions C2-C6 and adds any
violations found to the violation log for user’s reference.

\para If the violation log is empty, the RepOkValidator prints that
the predicate passed the test. This means that the \emph{RepOk} method
will not affect the output or the performance of the Korat search and
Korat will produce expected results.

\section{Example}
\label{sec:static-analysis-example}
In this section, we will be discussing the shortcomings of two
\emph{RepOk} methods written for the BinaryTree class shown in
Figure~\ref{fig:btreeDirectRepOk} in
Chapter~\ref{ch:shortcomings-of-korat}. The first example will see a
faulty \emph{RepOk} implementation that may affect the valid
structures output by the Korat search as it violates one or more of
conditions C1-C3. The second example will see an inefficient
\emph{RepOk} implementation that may affect the efficiency of the
Korat search as it violates one or more of conditions C4-C6. We will
also see the violation logs output by the RepOkValidator for both the
examples.


\subsection{Faulty predicate}
\label{sec:faulty-predicate}
Figure~\ref{fig:repOkKoratSatisfyCorrectness} shows a \emph{RepOk}
implementation that a user may write to generate non-isomorphic
BinaryTree structures. The \emph{RepOk} method may affect output of
Korat as it violates the conditions that need to be fulfilled for
Korat to generate the expected valid structures. To be specific, the
\emph{RepOk} implementation violates condition C3 from
Section~\ref{sec:static-analysis-conditions}. It violates C3 by using
the \texttt{MAX\_SCOPE\_PARAM} field from the \emph{Settings} class
(shown in Figure~\ref{fig:repOkKoratSatisfyCorrectness}). The
\texttt{MAX\_SCOPE\_PARAM} field may be changed during the Korat search
and this might make the Korat search output lesser number of valid
structures than expected.

\begin{figure}
\centering
\begin{lstlisting}[language=Java]
public class Settings {
    public static int MAX_SCOPE_PARAM = 3;
}

boolean repOKFaulty() {
    //makes sure the size is not more than limit
    if (size > Settings.MAX_SCOPE_PARAM) 
        return false;
    // checks that empty tree has size zero.
    if (root == null) return size == 0;
    Set visited = new HashSet();
    visited.add(root);
    LinkedList workList = new LinkedList();
    workList.add(root);
    // loop checks that the object graph is a tree.
    while (!workList.isEmpty()) {
        Node current = (Node) workList.removeFirst();
        if (current.left != null) {
            if (!visited.add(current.left))
                return false;
            workList.add(current.left);
        }
        if (current.right != null) {
            if (!visited.add(current.right))
                return false;
            workList.add(current.right);
        }
    }
    // checks that the size is consistent.
    return (visited.size() == size);
}

\end{lstlisting}
\caption{RepOk implementation that might make Korat output lesser number of structures as it doesn’t satisfy condition C3.}
\label{fig:repOkKoratSatisfyCorrectness}
\end{figure}

\para When the \emph{RepOK} predicate is analyzed using the
RepOkValidator, it prints a log of the violations that need to be
corrected by the user to ensure proper functionality of Korat search
when this \emph{RepOk} is used with
Korat. Figure~\ref{fig:repOkKoratSatisfyCorrectnessLog} shows the logs
printed by the RepOkValidator. Each log will mention the kind of the
violation, the line number at which the violation is found, the
predicate method inside which the violation is found and the class to
which the \emph{RepOk} method belongs to. Since all the violations
shown by Figure~\ref{fig:repOkKoratSatisfyCorrectnessLog} belong to
C1-C3, it is essential for the user to correct all the violations for
Korat to output the expected valid structures.

\begin{figure}
\centering
\begin{lstlisting}[language=Java]
- Analyzing repOkFaulty from edu.utexas.BinaryTree -
Condition C3 violated : usage of a static field - MAX_SCOPE_PARAM from class edu.utexas.BinaryTree$Settings at line 54 inside repOKEfficient in edu.utexas.BinaryTree
- repOkFaulty did not pass the test -
\end{lstlisting}
\caption{Violation log generated by the RepOkValidator when the faulty repOk is analyzed.}
\label{fig:repOkKoratSatisfyCorrectnessLog}
\end{figure}

\subsection{Inefficient predicate}
\label{sec:inefficient-predicate}
Figure \ref{fig:repOkMultipleBooleanVariables} shows another
\emph{RepOk} that the that a user may write to generate non-isomorphic
BinaryTree structures. This time the \emph{RepOk} does not affect the
output of Korat as it does not violate conditions C1-C3. Korat outputs
all valid non-isomorphic structures when the \emph{RepOk} method from
Figure~\ref{fig:repOkMultipleBooleanVariables} is used but, it also
evaluates way too many candidate structures to output them. This is
because the \emph{RepOk} implementation violates conditions C5 and
C6. Since violating conditions C4-C6 affects the efficiency of Korat,
it is observed that the execution of Korat with this predicate takes
more time compared to the predicate in Figure~\ref{fig:btreeDirectRepOk}, which passes the RepOkValidator test.

\para The \emph{RepOk} method violates C5 because the \emph{boolean}
variable \emph{acyclic} is \emph{defined} twice before it is
returned. Since the result of the predicate depends on the variable,
it is essential to return the result right after setting the value of
\emph{acyclic} for the first time. It also violated C6 as there are
multiple field references after setting the value of the variable
\emph{acyclic} and before it is actually returned as a part of the
result for the \emph{RepOk} method.

\begin{figure}
\centering
\begin{lstlisting}[language=Java]
boolean repOKMultipleVariables() {
    boolean sizeOk = true, acyclic = true;
    // checks that empty tree has size zero.
    if (root == null) sizeOk = (size == 0);
    Set visited = new HashSet();
    if(root != null) visited.add(root);
    LinkedList workList = new LinkedList();
    if(root != null) workList.add(root);
    // loop checks that the object graph is a tree.
    while (!workList.isEmpty()) {
        Node current = (Node) workList.removeFirst();
        if (current.left != null) {
            if (!visited.add(current.left)) 
                acyclic = false;
            else workList.add(current.left);
        }
        if (current.right != null) {
            if (!visited.add(current.right)) 
                acyclic = false;
            else workList.add(current.right);
        }
    }
    // checks that the size is consistent.
    return acyclic && sizeOk 
           && (visited.size() == size);  
}
\end{lstlisting}
\caption{RepOk that uses multiple boolean variables to decide its result. It violates C5 and C6 and ends up affecting the efficiency of Korat.}
\label{fig:repOkMultipleBooleanVariables}
\end{figure}

\para In the \emph{RepOk} shown in
F`igure~\ref{fig:repOkMultipleBooleanVariables}, values of multiple
boolean variables combine to form the result of the predicate. This is
very common for complex, recursive data structures like the
BinaryTree. In situations like this, the RepOkValidator proves to be
effective as it exactly points out the places at which the violations
are made. Figure~\ref{fig:repOkKoratSatisfyEfficiencyLog} shows the
violation log that gets generated when the \emph{RepOk} shown in
Figure~ \ref{fig:repOkMultipleBooleanVariables} is analyzed with the
RepOkValidator.

\begin{figure}
\centering
\begin{lstlisting}[language=Java]
- Analyzing repOKMultipleVariables from edu.utexas.BinaryTree -
Condition C5 violated : redefinition of field acyclic at line 91 inside repOKMultipleVariables in edu.utexas.BinaryTree.java
Condition C6 violated : usage of a field - left after defining return value from class edu.utexas.BinaryTree$Node at line 87 inside repOKMultipleVariables in edu.utexas.BinaryTree.java
Condition C6 violated : usage of a field - right after defining return value from class edu.utexas.BinaryTree$Node at line 89 inside repOKMultipleVariables in edu.utexas.BinaryTree.java
Condition C6 violated : usage of a field - right after defining return value from class edu.utexas.BinaryTree$Node at line 90 inside repOKMultipleVariables in edu.utexas.BinaryTree.java
Condition C6 violated : usage of a field - right after defining return value from class edu.utexas.BinaryTree$Node at line 92 inside repOKMultipleVariables in edu.utexas.BinaryTree.java
Condition C6 violated : usage of a field - size after defining return value from class edu.utexas.BinaryTree at line 96 inside repOKMultipleVariables in edu.utexas.BinaryTree.java
// More Logs ...
- repOKMultipleVariables did not pass the test -
\end{lstlisting}
\caption{Violation log generated by the RepOkValidator when the inefficient repOk is analyzed.}
\label{fig:repOkKoratSatisfyEfficiencyLog}
\end{figure}
