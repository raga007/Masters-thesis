\chapter{Library for predicates}
This chapter introduces a library that can be used by the programmer to write efficient predicates for Korat. Section 5.1 shows an example of using a method from the library that checks for acyclicity in a recursive data structure. The library presented in this chapter (\emph{korat.util.ReflectionLib}) currently has two methods :
\begin{itemize}
\item \emph{Set checkAcyclicity(Object o)} : checks if the given object is acyclic. It is assumed that this method is used when writing predicates for recursive data structures like BinaryTree. This method returns a \emph{java.util.HashSet} of the recursive objects if the \emph{Object o} is acyclic else, it returns a \emph{null}.
\item \emph{String serializeStructure(Object o)} : serializes the Object recursively, without accounting for the field accesses. Different levels in the provided object are represented with trailing \emph{tabs}. This method returns a \emph{String} that represents the serialized form of the \emph{Object o}.
\end{itemize}

\section{Example}
Figure 5.1 shows an example of how a user can use a library call to both reduce the size of \emph{repOk} and make sure that the \emph{repOk} is efficient. The methods provided in the library are written in such a way that they don't affect Korat's efficiency when used. In other words, they don't violate conditions C1-C6 shown in chapter 4. In figure 5.1, we use the \emph{checkAcyclicity} method to check the acyclicity property of BinaryTree data structure introduced in chapter 2. It is important to note that the \emph{repOk} shown in figure 5.1 has the same effect as the \emph{repOk} shown in figure 2.1 which is the most efficient version of a \emph{repOk} that can be written for the BinaryTree data structure. Using the \emph{checkAcyclicity} method, the user can avoid code bloat, as well as condition violations.

\begin{figure}
\centering
\begin{lstlisting}[language=Java]
boolean repOK() {
    // checks that empty tree has size zero.
    if (root == null) return size == 0;
    // checks that the object graph is a tree.
    Set nodeSet = ReflectionLib.checkAcyclic(root);
    // checks that the size is consistent.
    return (nodeSet != null && nodeSet.size() == size);
}
\end{lstlisting}
\caption{Binary tree example with repOk using checkAcyclic method from the library.}
\label{fig:btreeLibraryRepOk}
\end{figure}
