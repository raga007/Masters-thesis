\chapter{Library for predicates}
\label{ch:library-for-predicates}
This chapter presents a library that can be used by the programmer
to write short and efficient predicates for Korat. Section
\ref{sec:library-for-predicates-example} shows an example of using a
method from the library that checks for acyclicity in a recursive data
structure. The library presented in this chapter
(\emph{korat.util.ReflectionLib}) currently has two methods :
\begin{itemize}
\item \emph{Set checkAcyclicity(Object o)} : checks if the given
  object graph is acyclic. This method returns a
  \emph{java.util.HashSet} of the recursive objects if the
  \emph{Object o} is acyclic else, it returns a \emph{null}.
\item \emph{String serializeStructure(Object o)} : serializes the
  Object recursively, without accounting for the field
  accesses. Different levels in the provided object are represented
  with trailing \emph{tabs}. This method returns a \emph{String} that
  represents the serialized form of the \emph{Object o}.
\end{itemize}

\section{Example}
\label{sec:library-for-predicates-example}
Figure \ref{fig:btreeLibraryRepOk} shows an example of how a user can
use a library call to both reduce the size of \emph{RepOk} and make
sure that the \emph{RepOk} is efficient. The methods provided in the
library are written in such a way that they don't make unnecessary
field acesses that affect Korat's efficiency. In other words, they
don't violate conditions C1-C6 shown in chapter
\ref{ch:static-analysis}. In figure \ref{fig:btreeLibraryRepOk}, we
use the \emph{checkAcyclicity} method to check the acyclicity property
of our BinaryTree data structure introduced in chapter
\ref{ch:shortcomings-of-korat}. It is important to note that the
\emph{RepOk} implementation shown in figure
\ref{fig:btreeLibraryRepOk} has the same functionality as the
\emph{RepOk} implentation shown in figure \ref{fig:btreeDirectRepOk}
which is the most efficient version of a \emph{repOk} that can be
written for our BinaryTree data structure.

\begin{figure}
\centering
\begin{lstlisting}[language=Java]
boolean repOK() {
    // checks that empty tree has size zero.
    if (root == null) return size == 0;
    // checks that the object graph is a tree.
    Set nodeSet = ReflectionLib.checkAcyclic(root);
    // checks that the size is consistent.
    return (nodeSet != null && nodeSet.size() == size);
}
\end{lstlisting}
\caption{Binary tree example with repOk using checkAcyclic method from the library.}
\label{fig:btreeLibraryRepOk}
\end{figure}
